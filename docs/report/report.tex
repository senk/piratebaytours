\documentclass[12pt,a4paper,ngerman,english]{report}
\usepackage[utf8]{inputenc}
\usepackage[german]{babel}
\usepackage[draft]{todonotes} 
\usepackage{pdfpages}
\usepackage{pdflscape}
\usepackage{rotating}
\usepackage{epstopdf}
\usepackage{cite}
\usepackage{graphicx}
\usepackage{gensymb}
\usepackage{amsmath}
\usepackage{amssymb}
\usepackage{float}
\usepackage{fancyhdr}
\usepackage{textcomp}
\usepackage[numbers]{natbib}
\usepackage{varioref}
\usepackage{hyperref}
\usepackage[ngerman]{cleveref}
\usepackage{listings}

\usepackage[a4paper, top=30mm, left=30mm, right=30mm, bottom=30mm,headsep=10mm, footskip=12mm]{geometry}
\parindent0pt

%\usepackage[sorting=nyt, backend=biber, style=alphabetic]{biblatex}
%\addbibresource{mybib.bib}

\usepackage[onehalfspacing]{setspace}

\graphicspath{graphics}

\newcommand\tab[1][1cm]{\hspace*{#1}}

\pagestyle{fancy} %eigener Seitenstil
\fancyhf{} %alle Kopf- und Fußzeilenfelder bereinigen
\fancyhead[L]{\small{\leftmark}}

\newcommand{\icode}[1]{{\ttfamily{#1}}}

\renewcommand{\headrulewidth}{0.1pt} %obere Trennlinie
\fancyfoot[C]{\thepage} %Seitennummer
\setlength{\headheight}{13.6pt}

%\lstset{language=C++, basicstyle=\ttfamily\footnotesize, showstringspaces=false} %C++ code formatierung
\lstset{language=C++,
	basicstyle=\ttfamily\footnotesize,
	breaklines=true,
	%	literate={\ \ }{{\ }}1, 
	showstringspaces=false,
	keywordstyle=\color{blue}\ttfamily,
	stringstyle=\color{red}\ttfamily,
	commentstyle=\color{green}\ttfamily,
	morecomment=[l][\color{magenta}]{\#}
}

\newenvironment{abstractpage}
{\cleardoublepage\vspace*{\fill}\thispagestyle{empty}}
{\vfill\cleardoublepage}
\newenvironment{abstrac}[1]
{\bigskip\selectlanguage{#1}%
	\begin{center}\bfseries\abstractname\end{center}}
{\par\bigskip}

\begin{document}

%Titlepage
\begin{titlepage}
\begin{center}	
	\Huge{PirateBayTours}
	
	\line(1,0){400} \\
	\textsc{\Large Projekt im Modul Verteilte Informationsysteme WS2016/17}\\
	[10cm]

	{\Large\itshape Jesko Appelfeller \par Robin Naundorf \par Frederik Broer \par Jonas Droste \par  }
	%%{\small Martikelnummer: 65654}
	\vfill

	\large{betreut durch\par
		Prof. Dr.-Ing. Thomas Christian \textsc{Weik}}

	\vfill
	{\large \today\par}

\end{center}
\end{titlepage}


% Abstracts
\newpage
\begin{abstractpage}
	\begin{abstrac}{ngerman}
		\todo[inline]{Abstrakt schreiben}
	\end{abstrac}
	
	\begin{abstrac}{english}
		\todo[inline]{Abstrakt übersetzen}
	\end{abstrac}
\end{abstractpage}



\newpage
\tableofcontents

\chapter{Motivation und Anforderungen}
\todo[inline]{Motivation des Projektes\\
	Was waren seine Anforderungen?\\}

\chapter{Lösungskonzept}
\section{Architektur}
\todo[inline]{Aus welchen Komponenten besteht die Software?\\
	Was sind die Schnittstellen?\\}
\section{Backend}
\subsection{Django und Admin Interface}
\todo[inline]{Was bieted Django alles von Hausaus?}
\subsection{Database}
\todo[inline]{Wie ist die Datenbank aufgebaut?\\
	Welche Replikation setzen wir ein?\\
	Wie ist diese Implementiert?\\}
\section{Client}
\subsection{Java Client}
\todo[inline]{Welche Konzepte nutzt der Client? MVC\\
	Wie ist die Struktur im Client?\\
	Welche Klassen wurden implementiert?\\
	Wie erfolgt die Datenhaltung?}
\subsection{Local Cache DB}
\todo[inline]{Was wird lokal gespeichert?\\
}

\chapter{Geschäftslogik}
\todo[inline]{Wie werden Fälle wie Überbuchen gehandelt?\\
	Wie erfolgt die Replizierung von Server zu Client und zurück?\\
	Wir replizieren an zwei Punkten pro Tag\\}

\chapter{Fazit}
\todo[inline]{Was sind die Lessons Learned\\
	Was könnte man das nächste mal besser machen?\\
	Waren unsere ausgewählten Komponenten für das Problem geeignet?}


\bibliography{mybib}{}
\bibliographystyle{plainnat}



\end{document}
